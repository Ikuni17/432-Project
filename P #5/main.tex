\documentclass[11pt]{article}
\usepackage[margin=1in]{geometry}

\usepackage{hyperref}
\usepackage{multirow}
\usepackage{array}
\usepackage{amssymb}

%opening
\title{Cuckoo Hashing: P5, Progress}
\author{Group \#7 \\ 
\small{Matt Johnerson, Michael Shihrer, Bradley White}}

\begin{document}

\maketitle
\begin{center}
\renewcommand{\arraystretch}{1.5}
\begin{tabular}{| c | c | c | c |m{12.5cm}}
\cline{1-4}
\multicolumn{4}{| c |}{\large{Projected Timeline:}} \\ \cline{1-4}
{\bf Date} & {\bf Task} & {\bf Personnel} & {\bf Complete} \\ \cline{1-4}
Nov. 22 & Implement Cuckoo Hashing & Mike & \\ \cline{1-4}
Nov. 22 & Implement Perfect Hashing & Matt & \\ \cline{1-4}
Nov. 22 & Implement Bin Method & Brad & \\ \cline{1-4}
Nov. 22 & Design Experiment & All & \\ \cline{1-4}
Nov. 23 & P5 Due & All & \\ \cline{1-4}
Nov. 27 & Implement Experiment Driver & All & \\ \cline{1-4}
Nov. 28 & Code Review, Testing, \& Update & All & \\ \cline{1-4}
Dec. 1 & Run Experiment & All & \\ \cline{1-4}
Dec. 2 & Analyze Results & All & \\ \cline{1-4}
Dec. 2 & Design Video & All & \\ \cline{1-4}
Dec. 4 & Produce Video & All  & \\ \cline{1-4}
Dec. 5 & P6 Due & All &   \\ \cline{1-4}

\end{tabular}
\end{center}

\pagebreak
\begin{thebibliography}{9}
\bibitem{pagh}
Pagh and, Rodler.
\textit{Cuckoo Hashing}.
2001.
\bibitem{erlingsson}
Erlingsson, Manasse, and Mcsherry.
\textit{A Cool and Practical Alternative to Traditional Hash Tables}.
2006.
\bibitem{wiki}
\url{https://en.wikipedia.org/wiki/Perfect_hash_function}
\end{thebibliography}
\end{document}

